\documentclass[11pt]{book}

\usepackage[magyar]{babel}
\usepackage{hulipsum}
\usepackage{fancyhdr}
\usepackage{graphicx}
\usepackage{wrapfig}
\usepackage{amsmath}
\usepackage{amsfonts}
\usepackage{xcolor}
\usepackage{listings}

\title{Zárthelyi dolgozat\\ \small{A csoport}}
\author{Vitkolczi Dániel(XPGMCH)}

\newenvironment{vers}[2]{\begin{center}\textbf{#1}\\ \textit{#2}\\ \end{center}}{\vspace{2 pt}}

\lstdefinestyle{mystyle}{tabsize=2, numbers=right, stepnumber=2, showspaces=true}
\lstset{style=mystyle}

\begin{document}
%1. feladat
\maketitle
\pagestyle{fancy}
\fancyhead[RO]{\thesection}
\fancyhead[LE]{\thechapter}
\fancyfoot[RO, LE]{\thepage}
\fancyfoot[C]{}
\chapter{}
\section{szekció}
\hulipsum[1-2]
\section{szekció}
\hulipsum[3-4]

%2. feladat
\newpage
\chapter{}
\begin{wrapfigure}{1}{0.25\textwidth}
\centering
\includegraphics[height=4cm, scale=0.5]{szines}
\caption{perfect}
\setlength{\fboxsep}{0pt}\fbox{\includegraphics[height=4cm]{szepia}}
\caption{valami}
\end{wrapfigure}
\hulipsum[1-2]

%3. feladat
\newpage
\chapter{}
\textbf{1. Definició}
(sajátérték). Legyen $A\in{\mathbb{R}}^{nxn}$ négyzetes mátrix. Azt mondjuk, hogy 
$\lambda \in{\mathbb{C}}$ sajátérték és $v\in{\mathbb{C}^{n}}$ a $\lambda$ sajátértékéhez tartozó(jobb oldali) sajátvektora A-nak, ha \[Av = \lambda v\]
\textbf{2. Definició} (Karakterisztikus polinom). Jelölje $E\in{\mathbb{R}}^{nxn}$ az egység mátrixot. Az $A$ ún. \textit{karakterisztikus polinomja}
\[\varphi{(}\lambda{) := det(A}\color{red}{-}\lambda{E}\color{black}{) = }
\begin{vmatrix}
a_{11}\color{red}{-}\lambda\color{black} & a_{12} & \hdots & a_{1n} \\
a_{21} & a_{22}\color{red}{-}\lambda\color{black} & \hdots & a_{2n} \\
\vdots & \vdots & \ddots & \vdots \\
a_{n1} & a_{n2} & \hdots & a_{nn}\color{red}{-}\lambda
\end{vmatrix}\]
egy $n$-edikfokú polinom $\lambda$-ban.\\
\textbf{1. Tétel} (sajátérték meghatározása). Az $A\in{\mathbb{R}^nxn}$ mátrix sajátértékei az ún. \textit{karakterisztikus egyenlet}
\[\varphi{(}\lambda{) = 0}\]
megoldásai. Mivel a $\varphi{(}\lambda{)}$ \textit{karakterisztikus polinom} egy $n$-edikfokú polinom $\lambda$-ban, ezért komnplex számokon (multiplicitással együtt) $n$ megoldása van.

%4. feladat
\newpage
\chapter{}
\hulipsum[1-2]
\center Programkód 1. Bináris keresés C-ben
\setlength{\fboxsep}{0pt}\fbox{\lstinputlisting[language=C]{binsearch_it.C}}

%5. feladat
\newpage
\chapter{}
\begin{vers}{Valami}{valaki}
valami valami \\
valami és vlaami \\
köszönöm
\end{vers}
\end{document}